%Section: Work Experience at the top
\section{\texorpdfstring{\color{Blue}Expérience Professionelle}{Expérience Professionelle}}
\begin{longtable}{R{2.5cm}|p{14.8cm}}
  \emph{Aujourd'hui} & Architecte logiciel | Leader Technique, \textsc{Bull SAS}, France\\
  \textsc{Avril 2012} & 
  \vspace{-1em}
  \footnotesize{
    \begin{itemize}
      \item Reconstruction du dépôt logiciel de Bull Coriolis : réalisation, coordination et reporting
      \item Migration du serveur métier vers open cobol : suivi de projet et reporting
      \item Mise en place d'une base documentaire (wiki) et présentation aux 
      équipes de développement
      \item Mise en place de conventions de code
      \item Analyse et simplification de l'architecture existante
      \item Mise en place d'un framework de développement d'interface web (jquery, bootstrap, taglibs)
      \item Implémentation de nouvelles fonctionnalités du produit
      \item Implémentation, analyse et livraison de correctifs de bugs
    \end{itemize}
    \vspace{-1em}
  }\\&
 	\footnotesize{\emph{Technologies utilisées:} Apache Tomcat, JBoss RichFaces, jquery, Twitter Bootstrap, LESS, Hibernate, Spring MVC }\\
 \multicolumn{2}{c}{} \\
 	\textsc{Mars 2012} & Ingénieur Consultant, \textsc{Altran Technologies}, France\\
 	\textsc{Avril 2011}& \emph{IT Specialist} pour IBM, Software Solutions Center of Excellence\\&
 	\footnotesize{
 		Implémentation d'une solution de traçabilité et d'authentification
 	 	de produits pharmaceutiques pour un organisme public européen.
 		\begin{itemize}
			\item Modélisation des bases de données de 2 composants logiciels de la solution
			\item Conception et implémentation des applications web de 2 composants logiciels de la solution (JSF, JBoss Richfaces\ldots)
			\item Développement de Services Web
			\item Extension de la couche d'accès aux données
		\end{itemize}
		\vspace{-1em}
	}\\&
 	\footnotesize{\emph{Technologies utilisées:} Websphere Application Server 7, JBoss RichFaces, Infosphere Traceability Server, Hibernate }\\
 \multicolumn{2}{c}{} \\
 	\textsc{Avril 2011} & Ingénieur Consultant, \textsc{Altran Technologies}, France\\
 	\textsc{Mars 2011}& \emph{IT Specialist} pour IBM, Industry Solutions Insurance\\&
 	\footnotesize{Développement d'une application android dans le cadre d'une solution à destination 
 	des clients dans le domaine des assurances.}\\&
 	\footnotesize{\emph{Technologies utilisées:} Eclipse avec \emph{ADT}, Android}\\
 \multicolumn{2}{c}{} \\
\pagebreak[4]
 	\textsc{Février 2011} & Ingénieur Consultant, \textsc{Altran Technologies}, France\\
 	\textsc{Février 2010}& \emph{IT Specialist} pour IBM, Product Lifecycle Management Center of Excellence\\&
 	\footnotesize{
 		\begin{itemize}
 			\item Intégration d'ENOVIA V6, Oracle E-Business Suite et Maximo Asset Management grâce au middleware IBM (Websphere Process Server).
 			\item Intégration de PTC Windchill, Rational DOORS et IGE+XAO Electrical Expert grâce au middleware IBM (Websphere Process Server)
 		\end{itemize}
 		\vspace{-1em}
 	}\\&
 	\footnotesize{\emph{Technologies utilisées:} Websphere Integration Developer, Rational Software Architect, Websphere Business Modeler, Websphere Service Registry and Repository, 
 	Websphere Process Server, ENOVIA V6, Maximo Asset Management, Rational DOORS, Eclipse }\\
 \multicolumn{2}{c}{} \\
 	\textsc{Janvier 2010} & Ingénieur Consultant, \textsc{Altran Technologies}, France\\
 	\textsc{Décembre 2007}& \emph{IT Specialist} pour IBM, Sensor Solutions Center of Excellence\\&
 	\footnotesize{
 		\begin{itemize}
 		  \item Mettre en place le suivi et le contrôle des commandes et approvisionnements à l'aide de la RFID
 		  \item Implémentation d'une solution de suivi et d'authentification de containers pour \href{http://www.container-centralen.com/}{Container Centralen} (\href{http://www.container-centralen.co.uk/rfid/history.aspx}{description du projet}, \href{http://www.container-centralen.co.uk/rfid/user\%20guide\%20for\%20scanning.aspx}{guide d'utilisation})
 		  \item Modification et extension d'une solution existante de contrôle des interventions techniques dans un centre de données 
 	 	(Intégration avec Maximo Asset Management for IT, utilisation de la RFID pour contrôler l'installation des serveurs\ldots)
 	 	  \item Implémentation d'une solution de lutte contre la contrefaçon pour un fabriquant de vins et spiritueux (Utilisation de la RFID)
 	 	  \item Travail d'étude sur le protocole ONS
 	 	  \item Maintenance corrective et extension d'une plateforme M2M
 		\end{itemize}
 		\vspace{-1em}
 	}\\&
 	\footnotesize{\emph{Technologies utilisées:} DB2, Eclipse, IBM Infosphere Traceability Server, Lotus Expeditor, Eclipse, 
 	IBM Rational Software Architect, IBM Premises Server, Maximo Asset Management for IT, RFIDIC (Première version d'Infosphere Traceability Server - EPCIS) }\\
 \multicolumn{2}{c}{}\\
 	\textsc{Novembre 2007} & Ingénieur d'étude, \textsc{IBM}, France \\
 	\textsc{Février 2007}&
 	\footnotesize{
 	 	Implémentation d'une solution de paiement NFC sur téléphones portables (\href{http://www.nouvo.ch/s-007}{vidéo}) 
 	 	dans le cadre du projet Campus Nova pour le Crédit Agricole
 	 	\begin{itemize}
 	 	  \item Implémentation d'un porte monnaie électronique
 	 	  \item Intégration avec une plateforme de paiement en ligne
 	 	\end{itemize}
 	 	
	}\\&
	\vspace{-1em}
 	\footnotesize{\emph{Technologies utilisées:} J2ME, Java Card, DB2, Websphere Application Server}\\
\end{longtable}