\newcommand{\el}{\quad - \enspace}

%Section: Work Experience at the top
\section{\texorpdfstring{\color{Blue}Expérience Professionelle}{Expérience Professionelle}}
%\renewcommand{\labelitemi}{$\bullet$}
\begin{longtable}{R{2.5cm}|p{14.8cm}}
  \emph{Aujourd'hui}      & \textsc{Architecte logiciel | Développeur/Concepteur Senior JEE, CAFAT, Nouvelle-Calédonie}   \\
  \textsc{Avril 2014}     & \el Support et encadrement technique des équipes de développement                             \\
                          & \el Suivi, validation et intégration des développements externalisés                          \\
                          & \el Implémentation, analyse et livraison de correctifs de bugs sur les applicatifs métiers    \\
                          & \el Evolutions et corrections des bugs du framework de développement interne                  \\
                          & \el Rédaction des dossiers d'architecture en collaboration avec les architectes fonctionnels  \\
                          & \el Veille technologique                                                                      \\
                          & \footnotesize{\emph{Technologies utilisées:} JBoss EAP, IntelliJ Idea, Eclipse, Maven }       \\
 \multicolumn{2}{c}{}                                                                                                 \\
  \textsc{Mars 2014}      & \textsc{Architecte logiciel | Développeur/Concepteur Senior JEE, Bull SAS, France}            \\
  \textsc{Avril 2012}     & \el Reconstruction du dépôt logiciel de Bull Coriolis : réalisation, coordination et reporting\\
                          & \el Migration du serveur métier vers Open Cobol : suivi de projet et reporting                \\
                          & \el Solution documentaire collaborative (wiki) : mise en place et formation                   \\
                          & \el Evolutions et corrections : analyse, conception et développement                          \\
                          & \el Mise en place de conventions de code                                                      \\
                          & \el Mise en place d'un framework de développement d'interface web (jQuery, Bootstrap, taglibs)\\
                          & \footnotesize{\emph{Technologies utilisées:} Tomcat, Oracle DB, RichFaces, jQuery, Bootstrap, LESS, Hibernate, Spring, Eclipse, Maven }\\
 \multicolumn{2}{c}{} \\
 	\textsc{Mars 2012}      & \textsc{Ingénieur Consultant, Altran Technologies, France}\\
 	\textsc{Avril 2011}     & \emph{IT Specialist} pour IBM, Software Solutions Center of Excellence\\
                          &	Solution de traçabilité et d'authentification de produits pharmaceutiques pour EDQM (\href{https://www.edqm.eu/fr/eTACT-1466.html}{Projet eTACT}).\\
                          & \el Conception et développement d'applications web\\
                          & \el Base de données : Modélisation et implémentation de la couche ORM\\
                          & \el Conception et développement de Services Web \\
                          &	\footnotesize{\emph{Technologies utilisées:} \emph{WAS} 7, DB2, RichFaces, Infosphere Traceability Server, Hibernate, Ant}\\
 \multicolumn{2}{c}{} \\
 	\textsc{Avril 2011}     & \textsc{Ingénieur Consultant, Altran Technologies, France}\\
 	\textsc{Mars 2011}      & \emph{IT Specialist} pour IBM, Industry Solutions Insurance\\
                          &	Conception et développement d'une application Android pour tablette.\\
                          &	\footnotesize{\emph{Technologies utilisées:} Eclipse avec \emph{ADT}, Android}\\
 \multicolumn{2}{c}{} \\
 	\textsc{Février 2011}   & \textsc{Ingénieur Consultant, Altran Technologies}, France\\
 	\textsc{Février 2010}   & \emph{IT Specialist} pour IBM, Product Lifecycle Management Center of Excellence\\
                          & Mise en place d'un \emph{Enterprise Service Bus} (ESB) et moteur de Workflow\\
                          & \el Définition et implémentation de processus métiers\\
                          & \el Mise en place d'un ESB, implémentation d'un format pivot standard et définition de médiations\\
                          & \footnotesize{\emph{Technologies utilisées:} Websphere Integration Developer, RSA, Websphere Business Modeler, Websphere Service Registry and Repository, 
 	                          Websphere Process Server, ENOVIA V6, Maximo Asset Management, Eclipse }\\
 \multicolumn{2}{c}{} \\
 	\textsc{Janvier 2010}   & \textsc{Ingénieur Consultant, Altran Technologies, France}\\
 	\textsc{Décembre 2007}  & \emph{IT Specialist} pour IBM, Sensor Solutions Center of Excellence\\
                          & \el Mise en place du suivi et du contrôle des commandes et approvisionnements à l'aide de la RFID\\
 		                      & \el Projet de suivi et authentification de containers (\href{http://www.container-centralen.co.uk/rfid/history.aspx}{description}) : conception et développement\\
                          & \el Amélioration d'une solution de contrôle des interventions dans un centre de données (RFID)\\
                          & \el Solution de lutte contre la contrefaçon pour un producteur de spiritueux (RFID)\\
                          & \el Etude du protocole ONS : Analyse, \emph{POC}, documentation et présentation technique\\
                          & \el Maintenance corrective et évolutions d'une plateforme M2M (basée sur Websphere Portal)\\
                          & \footnotesize{\emph{Technologies utilisées:} DB2, Eclipse, Infosphere Traceability Server, Lotus Expeditor, Eclipse, 
 	                          Rational Software Architect, IBM Premises Server, Maximo Asset Management for IT, RFIDIC (Infosphere Traceability Server - EPCIS) }\\
 \multicolumn{2}{c}{}\\
 	\textsc{Novembre 2007}  & \textsc{Ingénieur d'étude, IBM, France} \\
 	\textsc{Février 2007}   &	Implémentation d'une solution de paiement NFC sur téléphones portables (\href{http://www.nouvo.ch/s-007}{vidéo}) 
 	 	                        dans le cadre du projet Campus Nova pour le Crédit Agricole\\
 	 	                      & \el Implémentation d'un porte monnaie électronique \\
                          & \el Intégration avec une plateforme de paiement en ligne\\
                          & \footnotesize{\emph{Technologies utilisées:} J2ME, Java Card, DB2, \emph{WAS}}\\
\end{longtable}
